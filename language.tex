% language.tex

\subsection{\foetus\ Syntax\label{sec:syntax}}
% syntax.tex

A \foetus\ program consists of {\em terms} and {\em definitions}.
\begin{center}
\begin{tabular}{rcll}
$P$ & $::$ &                       & {\it empty program} \\
    & $|$  & {\it term};       $P$ & {\it term to be evaluated} \\
    & $|$  & {\it definition}; $P$ & {\it definition for further use}
\end{tabular}
\end{center}
When processing input, \foetus\ evaluates the terms and stores the
definitions in the environment. ``Reserved words'' in the \foetus\
language are {\tt case}, {\tt of}, {\tt let} and {\tt in}.
%{\tt (}, {\tt )}, {\tt [}, {\tt ]}, {\tt \{}, {\tt \}}, {\tt |},
%{\tt .}, {\tt
Special characterss are {\tt ( ) [ ] \{ \} | .\ , ; = =>}. An identifier may
contain letters, digits, apostrophies and underscores.
If it starts with a small letter it stands for a variable,
else it denotes a constant.
\paragraph*{Term syntax.} In the following
$x$, $x_1$, $x_2$, $\dots$ denote variables,
$C$, $C_1$, $C_2$, $\dots$ constants and
$u$, $t$, $t_1$, $t_2$, $\dots$ \foetus\ terms.
\begin{center}
\begin{tabular}{rcll}
$t$ & $::$ & $x$       & {\it variable} \\
    & $|$  & $[x]t$    & {\it lambda} \\
    & $|$  & $t u$     & {\it application} \\
    & $|$  & $C(t)$    & {\it constructor} \\
    & $|$  & case $t$ of \{ $C_1 x_1 \darr t_1 | \dots | C_n x_n
      \darr t_n$ \}    & {\it pattern matching} \\
    & $|$  & $(C_1 = t_1, \dots, C_n = t_n)$
                       & {\it tuple} \\
    & $|$  & $t.C$     & {\it projection} \\
    & $|$  & let $x_1 = t_1, \dots, x_n = t_n$ in $t$
                       & {\it let} \\
    & $|$  & (t)       & {\it (extra parentheses) }
\end{tabular}
\end{center}

\paragraph*{Definitions.}
A definition statement has the form $x_1 = t_1, \dots, x_n = t_n$ (it
is a let-term without a ``body''). All variables $x_1, \dots, x_n$ are
defined simultanously, thus can refer to each other.

\paragraph*{Example.} The following \foetus\ program defines addition
on natural numbers (spanned by the two constructors {\tt O} ``zero'' and
{\tt S} ``successor'') and calculates $1+1$.

\begin{verbatim}
add = [x][y]case x of
        { O z => y
        | S x' => S(add x' y) };
one = S(O());
add one one;
\end{verbatim}

Note that although {\tt O} is a zero-argument-constructor the syntax
forces us to supply a dummy variable {\tt z} within the pattern
definition and also empty tuple {\tt ()} in the definition of {\tt one}.

%%% Local Variables:
%%% mode: latex
%%% TeX-master: "foetus"
%%% End:


\subsection{\foetus\ Type System}
In the following $x$, $x_1$, $x_2$, $\dots$ denote variables,
$C$, $C_1$, $C_2$, $\dots$ constants,
%$l$, $l_1$, $l_2$, $\dots$ labels,
$u$, $t$, $t_1$, $t_2$, $\dots$ \foetus\ terms,
$\tau$, $\sigma$, $\sigma_1$, $\sigma_2$, $\dots$ \foetus\ types and
$X$, $X_1$, $X_2, \dots$ type variables.
$\G = x_1:\sigma_1, \dots, x_n:\sigma_n$ denotes the context.
The judgement
$$
    \G \vdash t:\sigma
$$
means ``in context $\G$ term $t$ is of type $\sigma$''.
\pagebreak

\paragraph*{Type formation.}\nopagebreak[4]
\begin{center}
\begin{tabular}{rcll}
$\tau$ & $::$ & $\sigma \rightarrow \tau$
           & {\it $\rightarrow$-type} \\
    & $|$  & $\{ C_1 : \sigma_1 | \dots | C_n : \sigma_n \}$
           & {\it labeled sum type} \\
    & $|$  & $( C_1 : \sigma_1 , \dots, C_n : \sigma_n )$
           & {\it labeled product type} \\
    & $|$  & $\Pitype{ X } \tau$
           & {\it polymorphic type} \\
    & $|$  & $\tau \sigma$
           & {\it instantiation of polymorphic type} \\
    & $|$  & $\Let{X_1 = \sigma_1, \dots, X_n = \sigma_n}{\tau}$
           & {\it recursive type}
\end{tabular}
\end{center}
In the formation of a recursive type with Let $X_i$ may only appear
strict positiv in $\sigma_i$. We define congruence on types $\cong$ as
the smallest congruence closed under
$$
    \Let{\vec{X} = \vec{\sigma}}{\tau} \cong \tau[X_1 := \Let{\vec{X} =
\vec{\sigma}}{X_1}; \dots; X_n := \Let{\vec{X} =
\vec{\sigma}}{X_n}]
$$
($\vec{X}=\vec{\sigma}$ abbreviates $X_1 = \sigma_1, \dots, X_n =
\sigma_n$). Thus we can substitute congruent types:
$$
\ru{\G\vdash t:\sigma \quad \sigma\cong\tau}
   {\G \vdash t: \tau}
$$
For ploymorphic types we have rules like in System F:
$$
\rux{\G \vdash t : \sigma \quad \mbox{$X$ not free type variable in $\G$}}
    {\G \vdash t : \{X\}\sigma}
    {poly-i}
$$
$$
\rux{\G \vdash t: \{X\}\sigma}
    {\G \vdash t: \sigma[X:=\tau]}
    {poly-e}
$$


\subsection{Typing rules for \foetus\ terms}

We here only briefly introduce the typing rules. For more detailed
explanation, read a book about type theorie, e.g. \cite{NPS90}.

\paragraph*{Lambda abstraction.}
$$
\rux{\G,x:\sigma \vdash t:\tau}
    {\G \vdash [x]t : \sigma \rightarrow \tau}
    {\rightarrow-i}
$$

\paragraph*{Application.}
$$
\rux{\G \vdash t:\sigma\rightarrow\tau \quad \G \vdash u:\sigma}
    {\G \vdash t u : \sigma}
    {\rightarrow-e}
$$

\paragraph*{Constructor.}
\[
\rux{\G\vdash t:\sigma_i}
    {\G\vdash C_i(t) : \{ C_1 : \sigma_1 | \dots | C_n : \sigma_n \}}
    {\{\}-i}
\]

\paragraph*{Pattern matching.}
\[
\rux{\G\vdash t: \{ C_1 : \sigma_1 | \dots | C_n : \sigma_n \}\qquad
     \G,x_i:\sigma_i \vdash u_i : \sigma \ \mbox{for all $1 \leq i
       \leq n$}}
    {\G\vdash \case{t}{C_1(x_1)\darr u_1 | \dots | C_n(x_n)\darr u_n}
      : \sigma}
    {\{\}-e}
\]

\paragraph*{Tupels.}
\[
\rux{\G\vdash t_i : \sigma_i\ \mbox{for all $1\leq i\leq n$}}
    {\G\vdash (C_1=t_1,\dots,C_n=t_n) : ( C_1 : \sigma_1 , \dots ,
  C_n : \sigma_n )}
    {()-i}
\]

\paragraph*{Projection.}
\[
\rux{\G\vdash t:( C_1 : \sigma_1 , \dots , C_n : \sigma_n )} {\G\vdash t.C_i
  : \sigma_i} {()-e}
\]

\paragraph*{Let.}
\[
\rux{\G,x_1:\sigma_1,\dots,x_n:\sigma_n \vdash t_i : \sigma_i \
  \mbox{for all $1\leq i\leq n$} \qquad \G,x_1:\sigma_1,\dots
  ,x_n:\sigma_n \vdash u : \tau} {\G\vdash \xlet{x_1=t_1;\dots
    ;x_n=t_n}{u}:\tau} {let}
\]
In \foetus\ type checking is not yet implemented and it is assummed
that all terms entered are well typed. Of course, only for well typed
terms the termination check produces valid results.
\paragraph*{Example.} The following well-known example for
non-termination passes the \foetus\ termination checker, but it is not
well typed.
\begin{verbatim}
f = [x]x x;
a = f f;
\end{verbatim}
\foetus\ output:\nopagebreak
\begin{verbatim}
f passes termination check
a passes termination check
\end{verbatim}
\runexample{f+\%3D\%5Bx\%5Dx+x\%3B\%0D\%0Aa+\%3D+f+f\%3B\%0D\%0A}

%%% Local Variables:
%%% mode: latex
%%% TeX-master: "foetus"
%%% End:
